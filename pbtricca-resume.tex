\documentclass[letterpaper,11pt]{article}
\usepackage[pagesetup]{tucv}
\usepackage[colorlinks=true,urlcolor=black]{hyperref}
\fancyfoot[C]{Philip Tricca - \thepage}

\begin{document}

% renew resjob command to get rid of weird line break in date range
\RenewDocumentCommand\resjob{ommm}{
    \resentry[10pt]{
    \setlength{\parskip}{1ex plus 0.5ex minus 0.2ex} \textbf{#2}
    \IfNoValueTF{#1}
    {}
    {\newline #1}
    }{\mbox {#3 - #4}}
}

% Page heading and name/contact info table
\begin{tabular*}{7in}{l@{\extracolsep{\fill}}r}
\textbf{\Large Philip B. Tricca}
& flihp@twobit.us \\
San Bruno CA, 94066
& http://twobit.us \\
\end{tabular*}

    \resheading {Qualification Summary}
Computer Engineer with 7 years experience working in a government research
environment and 6 years analyzing, designing  and building security focused
Linux-based products and tools. Strong technical, organizational and planning
skills. Professional, trustworthy and a natural communicator. Dedicated to
technical excellence in software design, implementation and documentation
with a focus on security.

    \resheading {Professional Experience}
    \resemployer {Intel}{\mbox{Santa Clara, CA}}
    \resjob [
      Performed design and development tasks aimed at promoting the adoption and proper use of platform security technologies.
      Primary tasks include:
    ]{Platform Architect, Platform Security Devision}{10/2014}{present}
    \begin {itemize}
      \setlength {\itemsep}{1pt}
      \setlength {\parskip}{0pt}
      \setlength {\parsep}{0pt}
      \item Design and implementation of TPM2 Software Stack (TSS2) APIs and
            plumbing. All API design done collaboratively with the Trusted
            Computing Group (TCG) TSS2 working group. Source code implementing
            the TSS2 libraries, system daemon and tools based on TCG
            specifications published as OSS:
            \url{https://github.com/tpm2-software/}
      \item Identification of high value usage models for Intel\textsuperscript{\textregistered} SGX.
            A specific focus in this task is analysis and documentation of threat models and impact to system security.
            These designs and documentation were heavily peer reviewed and have moved on to prototype development.
    \end {itemize}
    \resemployer {Citrix Systems}{\mbox{Bedford, MA}}
    \resjob [
      The role of architect is a super-set of the responsibilities of a software engineer.
      Additional responsibilities included:
    ]{Architect, Client Virtualization Group}{10/2013}{07/2014}
    \begin {itemize}
        \setlength {\itemsep}{1pt}
        \setlength {\parskip}{0pt}
        \setlength {\parsep}{0pt}
      \item Cross-product design and requirements gathering to reduce duplication, standardize implementations and reduce likelihood of flaws.
      \item Architect and implement virtual disk encryption, lead customer requirements and design review.
    \end {itemize}
    \resjob [
      Developer on XenClient XT product: a client hypervisor focused on strong separation and security properties.
      Primary customers include government defense agencies.
      Contributions to this project include:
    ]{Senior Software Engineer, Client Virtualization Group}{08/2011}{10/2013}
    \begin {itemize}
        \setlength {\itemsep}{1pt}
        \setlength {\parskip}{0pt}
        \setlength {\parsep}{0pt}
      \item Design and implementation of platform measured launch using the TPM and Intel\textsuperscript{\textregistered} TXT.
      \item Defining and documenting threat models and the software architectures that defend against them.
      \item Development of platform mandatory access control policy (SELinux / XSM).
      \item Collecting requirements from customers and evaluators / testers for the purpose of knowledge transfer and certification.
    \end {itemize}
    \resemployer {Air Force Research Lab}{\mbox{Rome, NY}}
    \resjob [
      Performed a mix of work on research efforts relevant to military needs in the area of information systems security and contract management.
    ]{Computer Engineer, Information Directorate}{07/2004}{08/2011}
      \begin {itemize}
        \setlength {\itemsep}{1pt}
        \setlength {\parskip}{0pt}
        \setlength {\parsep}{0pt}
      \item Lead Engineer on SecureView project. Responsible for security architecture, technical design and customer reviews.
      \item Design improvements to ``Guard'' systems through the application of mandatory access control mechanisms and information flow principles.
      \item Maintain relationships with security community through academic and industry conferences.
      \item Contract management and technical guidance of cooperative research agreements with academic institutions including Cornell and PSU totaling over \$2M.
      \end {itemize}
      \resemployer {Open Source}{}
      \resjob [
        Contributing code, testing and/or debugging to open source projects.
        Blogger and active participant in open source and security community.
      ] {Contributor} {01/2010} {present}
      \begin {itemize}
        \setlength {\itemsep}{1pt}
        \setlength {\parskip}{0pt}
        \setlength {\parsep}{0pt}
      \item
        Creator and maintainer of the meta-measured OpenEmbedded layer to automate measured launch in embedded-style build environment.
        Currently listed in the official OE layer index.
      \item Contributor to the meta-selinux and meta-virtualization OpenEmbedded layers.
      \item
        Dual purpose work on Masters Project and XenClient XT:
        Integration of SELinux with XenClient management stack to separate mutually distrusted QEMU processes (sVirt architecture).
      \end {itemize}

    \resheading{Education}
    \begin{itemize}
    \item[]
      \resschool{Syracuse University}{New York}
      \resdegree[
        Software Security and Systems Assurance\\
        Advisor: Dr. Wenliang (Kevin) Du\\
        Cumulative GPA: 3.85
      ]{M.S.}{Computer Engineering}{2006-2012}
      \resdegree[
        Focus of study: Software Engineering.\\
        Particular interest in system level programming (i386 Assembly, C).\\
        Cumulative GPA: 3.78
      ]{B.S.}{Computer Engineering}{2000-2004}
    \end{itemize}

    \resheading{Awards}
    \begin {itemize}
      \setlength {\itemsep}{1pt}
      \setlength {\parskip}{0pt}
      \setlength {\parsep}{0pt}
    \item Key Contributor Award.
      Trusted Computing Group.
      2017-11-01
    \item Outstanding Engineering Professionalism Award.
      Mohawk Valley Engineers Executive Council.
      2010-02-18
    \item Special recognition of service from the Commander for service rendered while leading Commander Directed Investigation, 2009.
    \end {itemize}

    \resheading {Public Speaking \& Publications}
    \begin{itemize}
    \item {\it TPM2 Software Stack: Device Driver to Event-Driven Applications}
    Linux Plumbers Conference
    Los Angeles, California.
    2017
    URL: \url {http://www.linuxplumbersconf.org/2017/ocw/sessions/4675.html}
    \item {\it Securing Embedded Linux Systems with TPM2}.
    Embedded Linux Conference
    Portland, Oregon.
    2017.
    URL: \url {https://www.youtube.com/watch?v=0qu9R7Tlw9o}
    \item {\it In-Guest Mechanisms to Strengthen Guest Separation}.
    Xen Summit colocated with LinuxCon.
    Edinburgh, Scotland,
    2013.
    URL: \url {https://www.youtube.com/watch?v=6Q8mlTBn-ZI}
    \item {\it A pipeline development toolkit in support of secure information flow goals}.
    Proceedings of the 5th Annual CSIIR Workshop : Cyber Security and Information Intelligence Challenges and Strategies.
    Oak Ridge, Tennessee.
    2009.
    \end{itemize}
    Representative writing samples:
    \begin {itemize}
      \setlength {\itemsep}{1pt}
      \setlength {\parskip}{0pt}
      \setlength {\parsep}{0pt}
    \item
      TCG TSS2 async APIs and event driven programming:
      \url {https://blog.twobit.us/2017/10/15/tcg-tss2-async-api-event-driven/}
    \item
      TrEE vs TCG: Dueling TPM2 Specs:
      \url {http://blog.twobit.us/2015/11/tree-vs-tcg/}
    \item
      Attaching a TPM on the LPC:
      \url {http://blog.twobit.us/2015/02/attaching-tpm-on-lpc/}
    \item
      Measured launch on OE-Core:
      \url {http://blog.twobit.us/2013/01/meta-measured/}
    \item
      An on-going series documenting my masters project implementing the sVirt architecture on XenClient:
      \url {http://blog.twobit.us/tag/mastersproject/}
    \item
      A two part series describing the ``why'' and ``how'' of configuring a network driver domain on Xen:
      \url {http://blog.twobit.us/2010/07/xen-network-driver-domain-wh/}
      \url {http://blog.twobit.us/2010/07/xen-network-driver-domain-how/}
    \end {itemize}

    \resheading {Relevant Technical Proficiencies}
    \begin {itemize}
      \setlength {\itemsep}{1pt}
      \setlength {\parskip}{0pt}
      \setlength {\parsep}{0pt}
      \item TPM2 \& TCG TSS2 API and infrastructure concepts and implementation
          on Linux and Windows.
      \item GNU Autotools build and test harness.
      \item Test strategies and test driven development concepts.
      \item Continuous integration and test automation: travis-ci, coveralls,
          cmocka etc.
      \item Open Source Software development and project management /
          maintainership with 3 years experience maintaining large C code
          base.
      \item Proficiency with Open Embedded build system and the construction of meta layers / distros.
      \item Flask policy development and system configuration (SELinux / XSM).
      \item Appropriate usage of the TPM for constructing measured software systems.
      \item Threat modeling software systems using the STRIDE methodology.
      \item Appropriate usage of OpenSSL APIs (EVP/RAND).
      \item Programming in popular languages (shell,C,Python), quickly learns new languages as needed.
      \item Professional / technical writing and documentation suitable for submission as project deliverables.
      \item Proficiency with popular OSS tools and work flows.
      \item Document preparation in \LaTeX.
    \end {itemize}
\end {document}
