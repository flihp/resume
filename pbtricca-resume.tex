\documentclass[letterpaper,11pt]{article}
\usepackage[pagesetup]{tucv}
\usepackage[colorlinks=true,urlcolor=black]{hyperref}
\fancyfoot[C]{Philip Tricca - \thepage}

\begin{document}

% Page heading and name/contact info table
\begin{tabular*}{7in}{l@{\extracolsep{\fill}}r}
\textbf{\Large Philip B. Tricca}
& 978.766.5515 \\
13 Highland Ave. \#1
& flihp@twobit.us \\
Somerville MA, 02143
& http://twobit.us \\
\end{tabular*}

% qualification summary
    \resheading {Qualification Summary}
    Computer Engineer with 7 years experience working in a government research environment and 1 year corporate.
Strong technical, organizational and planning skills.
Professional, trustworthy and all-around team player.
Dedicated to technical excellence in software design, implementation and documentation with a specific focus on security.
    % relevant experience
    \resheading {Professional Experience}
    \resemployer {Citrix Systems}{Syracuse,NY}
    \resjob [
      Developer on XenClient XT product.
      XT is a variant of the XenClient client hypervisor with particular focus on strong separation and security.
      Primary customers include U.S. and foreign government defense agencies.
      Responsibilities on this project include:
    ]{Senior Software Engineer}{2011}{present}
    \begin {itemize}
        \setlength {\itemsep}{1pt}
        \setlength {\parskip}{0pt}
        \setlength {\parsep}{0pt}
      \item Development of the platform SELinux and XSM policy.
      \item Collecting requirements from customers and evaluators / testers for the purpose of system improvements and knowledge transfer.
      \item Supporting early pilot programs with architectural, debugging and prototyping services.
    \end {itemize}
    \resemployer {Air Force Research Lab}{Rome,NY}
    \resjob [
      Performed a mix of work on research efforts relevant to military needs in the area of information systems security and contract management.
    ]{Computer Engineer}{2004}{2011}
      \begin {itemize}
        \setlength {\itemsep}{1pt}
        \setlength {\parskip}{0pt}
        \setlength {\parsep}{0pt}
      \item Worked closely with Citrix engineers to develop and integrate SELinux support in XenClient product.
      \item Familiarity with Common Criteria Protection Profiles, NIST 800-53, CNSSI 1253.
      \item Worked on design improvements to ``Guard'' systems through the application of Mandatory Access Control (MAC) and general information flow principles.
      \item Developed and maintained relationships with computer security community through academic conferences and Linux Security Summit (SELinux).
      \item Contract management and technical guidance of cooperative research agreements (IARPA funded) with academic institutions including Cornell and PSU totaling over \$2M.
      \end {itemize}
      \resemployer {Open Source}{}
      \resjob [
        Contributed code, testing and/or debugging to open source projects.
        Blogger and Twitter user, actively participating in open source and security community.
      ] {Contributor} {2010} {present}
      \begin {itemize}
        \setlength {\itemsep}{1pt}
        \setlength {\parskip}{0pt}
        \setlength {\parsep}{0pt}
      \item Dual purposed work on Masters Project and work on XenClient XT: integrating mandatory access control (SELinux) with XenClient management stack to separate mutually distructing QEMU processes (sVirt architecture).  All code and documentation is open source.
      \item Some contributions to Debian Squeeze SELinux policy.
      \end {itemize}
      \pagebreak
    % education
    \resheading{Education}
    \begin{itemize}
    \item[]
      \resschool{Syracuse University}{New York}
      \resdegree[
        Software Security and Systems Assurance\\
        Advisor: Dr. Wenliang (Kevin) Du\\
        Cumulative GPA: 3.85
      ]{M.S.}{Computer Engineering}{2006 to present}
      \resdegree[
        Focus of study: Software Engineering.\\
        Particular interest in system level programming (i386 Assembly, C).\\
        Cumulative GPA: 3.78
      ]{B.S.}{Computer Engineering}{2000 to 2004}
    \end{itemize}

    % awards
    \resheading{Awards}
    \begin {itemize}
      \setlength {\itemsep}{1pt}
      \setlength {\parskip}{0pt}
      \setlength {\parsep}{0pt}
    \item Outstanding Engineering Professionalism Award.
      Mohawk Vally Engineers Executive Council.
      2010-02-18
    \item Special recognition of service from the Commander for service rendered while leading Commander Directed Investigation, 2009.
    \end {itemize}

    % publications
    \resheading {Publications}
    {\it A pipeline development toolkit in support of secure information flow goals}.
    Proceedings of the 5th Annual CSIIR Workshop : Cyber Security and Information Intelligence Challenges and Strategies.
    Oak Ridge, Tennessee.
    2009.\\
    Popular blog posts:
    \begin {itemize}
      \setlength {\itemsep}{1pt}
      \setlength {\parskip}{0pt}
      \setlength {\parsep}{0pt}
    \item
      An on-going series documenting my masters project implementing the sVirt architecture on XenClient:
      \url{http://twobit.us/blog/tag/mastersproject/}
    \item
      A two part series describing the ``why'' and ``how'' of configuring a network driver domain on Xen:
      \url{http://twobit.us/blog/2010/07/xen-network-driver-domain-wh/}
      \url{http://twobit.us/blog/2010/07/xen-network-driver-domain-how/}
    \end {itemize}
    % technical proficiencies
    \resheading {Relevant Technical Proficiencies}
    \begin {itemize}
      \setlength {\itemsep}{1pt}
      \setlength {\parskip}{0pt}
      \setlength {\parsep}{0pt}
      \item SELinux policy development and system configuration.
      \item XSM concepts and policy development.
      \item Design and implementation of secure systems and Cross Domain Solutions.
      \item Familiarity with OpenEmbedded build system and Angstrom Distro.
      \item Programming in various languages (shell,C,C++,C\#), quickly learns new languages as needed.
      \item Basic git usage.
      \item Pipeline processing \& system design for deployment on SELinux systems.
      \item Document preparation in \LaTeX.
    \end {itemize}
\end {document}
